\section{Design and Implementation}
 Three parts are combined to make the Haystack system, it is
 Haystack Directory, Haystack Store and Haystack Cache.
 Each one of them serving its own purpose. The Directory allowing for
 simplifications when finding the image on the actual servers containing the
 image. That server part is the Haystack Store, it contains the actual data, and
 allows for reading, writing and deleting. The cache is a customized cache so that
 it is very fast to find files that is newly uploaded.

\subsection{Haystack}
The haystack is designed to achieve four main goals:
\begin{itemize}
  \item High throughput and low latency
  \item Fault-tolerant
  \item Cost-effective
  \item Simple
\end{itemize} 

\subsubsection{Haystack Directory}
  What the Haystack Directory provides is mappings between physical and logical
  volumes, this mapping is then used when constructing a url for image request.
  The Directory can also determine if the request is handled by CDN or by the 
  Haystack Cache directly. 
  
\subsubsection{Haystack Store}
 The Haystack Store maintains persistent storage for images. On request it returns
 either the photo or an error if the file not found. \\
 A data structure of maps between keys and images are kept in-memory to retrieve
 the images as fast as possible. The data structure includes data such as offset,
 size ,flags, a cookie with a random number, as well as the corresponding key. 
 Haystack Store allows for three operations: read, write and delete.\\
 To read a picture from the Store the key is read, file is thereby found by the mappings
 and the cookies random number is checked as well as the integrity of the file, and then
 returned. To write a picture to the Haystack Store, the Haystack web servers will provide
 necessary values for the mappings and then append it to the disk.
 Delete is simple, just set flag to notify that the file is deleted, and the delete
 is handled later.
  
\subsubsection{Haystack Cache}
  The cache is organized as a distributed hash table, and the photo id is used as key.
  If it is found in cache it will respond to either CDN or user, depending on who made
  the request. A request from CDN will not be cached, and neither will a request from a
  full Store machine. The reason for this is that if they would cache CDN the reason for
  the Haystack Cache will be lost, hence that was one of the reasons for using something
  else. Also, if a request is coming from a full Store machine, because it has been proven
  that the most heavily accessed photos is they who were just uploaded, hence from a write 
  enabled Store.