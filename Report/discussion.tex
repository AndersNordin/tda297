\section{Discussion}
The system can be simplified to a sequencer with a number of storagenodes.
The sequencer beging the Haystack Directory makes sure the requests get the 
proper id, and the correct node get the correct information. 
\subsection{Fault-tolerance}
They talk about fault-tolerance, but not much about how to solve faulty messages, mostly
about faulty drives. It is assumed that everything is working, and a system named pitchfork
is proactivly checking for failures. An other aproach could be to build a system where this
is not needed. It is a way to make sure that messages are not sent to faulty entities. 

An alternate solution would be to try to find or mask errors by using voting. 

Replication across servers is not mensioned in the paper, only RAID 6. It is probable that
they have multiple replicas to adress performance issues when server is located
long physical distances from users. In the paper is mension that the Haystack Store 
has a system for loadbalancing which is indicating that they store on muliple servers
at once. Although, it might be a good idea to not replicate to much, hence it is uploaded
~60TB weekly. That means hughly costs just to have the RAID 6 repliciation to handle
individual disc failures. 


\inlinetodo{Possibility to recreate in-memory maps on store}
\inlinetodo{What are our thoughts about it? Good/Bad?}
