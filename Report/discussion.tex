\section{Discussion}
The system can be simplified to resemble a sequencer connected to a number of storage nodes.
The sequencer, the Haystack Directory, makes sure the requests get the
proper id, and the user gets the correct information for their image. When the user
is fetching the actual data from the Storage node, it uses the id from the sequencer
to find the demanded information. With this simplification in mind, we can reason about how it
functions.
\subsection{Fault-tolerance}
They talk about fault-tolerance, but not much about how to solve faulty messages, they talk
mostly about faulty drives. It is assumed at all time, that everything is working, and a system
named pitchfork is proactively checking for failures. An alternate solution would be to
try to find or mask errors by introducing voting.

Replication across servers is not mentioned in the paper, only RAID 6. It is probable that
they have multiple replicas to address performance issues when server is located
long physical distances from users. In the paper is mentioned that the Haystack Store
has a system for load balancing which is indicating that they store the same on multiple servers.
Although, it might be a good idea to not replicate too much, hence it is uploaded
\~60TB weekly. That means highly costs just to have the RAID 6 replication to handle
individual disc failures.
% Read and correct /A