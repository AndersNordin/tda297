

\documentclass{article}
\title{Lab3}
\author{anordin@student.chalmers.se\quad Anders Nordin\\
        viklin@student.chalmers.se\quad Viktor Lindstr\"{o}m}
\date{\today}
\usepackage[T1]{fontenc}
\usepackage[utf8]{inputenc}
\usepackage[english]{babel}
\usepackage{verbatim}
\usepackage{enumerate}
\usepackage{steinmetz}%för att få tillgång till /phase
\usepackage{amsmath}%massa trevliga symboler
\usepackage{siunitx}%enklare notation på enheter
\usepackage{tikz}
\usepackage{newclude}
\usepackage{setspace}
\usepackage{todonotes}
\usepackage{titlesec}
\newcommand{\inlinetodo}[2][]{\todo[caption={#2},inline,#1]{#2}}
\newcommand{\checknote}[2][]{\todo[caption={#2},size=\small,color=yellow!40,#1]{\begin{spacing}{0.5}#2\end{spacing}}}

\usetikzlibrary{arrows,decorations.pathmorphing,backgrounds,positioning,fit,petri}
\usepackage{fullpage}
\begin{document}
\maketitle
\newpage

\section{Introduction}
  This report will discuss and try to find a good solution in finding a 
  good way to route a message through a sensor-network.
  First a given basic solution will be given, after that
  we will present our findings. Each solution is given with first
  a breif description of how the algorithm find the route. Then 
  result of the solution is given by five runs of two given topologies,
  Grid and Comb. Grid topology is a totaly symetric grid, 4 * 10 nodes.
  the Comb is the same as the previous grid, but with the exception of
  a couple of nodes taken out, making it a rather asymetric graph.
  Each algorithm is tested five times in each grid.
\part{Improve the basic routing}
\section{Basic solution}
  The given program creates its route by the following algorithm.
  During anouncement phase, if the node has not already chosen an 
  node to route its messages through and if it is shorter
  from that node to the sink, use that node as next node.
  \subsection{Result}
    \begin{tabular}{c|c}
      Grid topology & Comb topology\\
      \hline
      \hline
      161 & 112\\
      169 & 103\\
      141 & 91\\
      159 & 118\\
      146 & 112\\
      \hline
      155.2&107.2\\
    \end{tabular}

  \subsection{Discussion}
    One of the problem is that it just choses the first one,
    so result will vary between each execution.
    If unlucky the nodes with distance MAXDISTANCE from
    sinknode will send its messages all the way to sink
    without any relay, because the sink is the first one
    send a announcement to them, and then they will relay
    all their messages to it, causing them to run out of 
    battery quickly. If the neighbouring nodes
    are doing the same thing, it will be a strip of dead
    nodes causing other nodes to not be able to send messages
    because there is no way to relay messages. 
\section{Frist try, Battey important}
  The first attempt was to just use the battery of the other nodes
  important, nothing else is important. If a sending node has alot of battery
  send it to them, and if they have low battery, skip them.
  \subsection{Result}
    \begin{tabular}{c|c||c|c}
      Grid topology & Comb topology & Basic grid & Basic comb\\
      \hline
      \hline
      115 & 95 & 161 & 112\\
      131 & 68 & 169 & 103\\
      89  & 60 & 141 & 91\\\
      81  & 70 & 159 & 118\\
      93  & 83 & 146 & 112\\
      \hline
      101.8&75.2&155.2&107.2\\

      
    \end{tabular}
  \subsection{Discussion}
    The battery itself does not seem to be a good idea by observing the results. 
    Although the messages will be divided between the nodes depending on the battery,
    so the nodes will die approximatly the same time, it will not cause the 
    system to be allowed to send more messages. Probably because messages will be sent 
    a long way, when not necessary, causing the node to spend alot of battery on 
    every single message.
    It does not take in to account to send messages in the wrong way.
  \section{Second try, Battery important}
    The previous just took the battery in to account, what if it also had
    some notion of what was the "right way", so it could not send messages
    back down in the graph. The "right way" being up or left, hence the sink
    node is located in the top left corner.
  \subsection{Result}
    \begin{tabular}{c|c||c|c}
      Grid topology & Comb topology & Basic grid & Basic comb\\
      \hline
      \hline
      156 & 123 & 161 & 112\\
      166 & 113 & 169 & 103\\
      178 & 123 & 141 & 91\\\
      164 & 111 & 159 & 118\\
      169 & 108 & 146 & 112\\
      \hline
      166,6&115.6&155.2&107.2\\
    \end{tabular}
    \section{Discussion}
      This seems to be preforming alot better, which was suspected. But still not a huge
      improvement. This tells us that the distance is more importat factor than battery
      in choosing node to send to. Although, it is a little better than the Basic algorithm
      so it still is not to be ignored.
\part{Clustered data aggregation}
  \end{document}
