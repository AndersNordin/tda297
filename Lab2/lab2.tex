\documentclass{article}
\title{TDA297 Laboration 2}
\author{?@student.chalmers.se\quad Anders Nordin\\
        viklin@studnet.chalmers.se\quad Viktor Lindstr\"{o}m}
\usepackage{circuitikz}
\date{\today}
\usepackage[T1]{fontenc}
\usepackage[utf8]{inputenc}
\usepackage[swedish]{babel}
\usepackage{verbatim}
\usepackage{enumerate}
\usepackage{steinmetz}%för att få tillgång till /phase
\usepackage{amsmath}%massa trevliga symboler
\usepackage{siunitx}%enklare notation på enheter
\usepackage{fullpage}
\begin{document}
\maketitle
\newpage
\section{Introduction}
  This report describes some of the problems of distributed network, and how to maintain
  total order aswell as causal order in a system that has reliable unicast FIFO order messagaging.\\
  The report will start with a short decription of the system and what algorithms 
  used. Then reliability is descussed and how the system solve integrity, validity and
  agreement. After that an explanation of how the system holds for total and causal order.
  Then there is an explination of how the system handle crashing processes. Lastly there
  is a conclusion.
\section{Description of system}
  The system consists of N processes that communicate with each other by TCP channels.
  TCP channels gives the system reliable unicast with FIFO order.Upon this system a 
  reliable total order and causal order multicast is implemented. In this system there
  exsists a single node that is also a sequencer. All messages will pass through the sequencer
  and the sequencer will add a timestamp to that message, and then distribute it to all nodes.
  The nodes will then receive the message and then start flooding it to all other nodes.
  In order to create the correct order, the messages is not delivered to the node directly,
  but is put in a queue untill the correct sequence number is aquired.
  \subsection{Assumptions made}
  \label{assumption}
  \begin{itemize}
  \item Reliable unicast
  \item FIFO-order unicast
  \item No partitioning in the network
  \item No link-failures
  \end{itemize}
\section{Reliability}
  %Short description here, proofs in subsections!
  The reliability is provided by fooding the message. By the assumptions made in \ref{assumption} we can rely on 
  that each single message sent by a working process $P_i$ to its neighbours $P_j$ to $P_n$ will be delivered.
  And hence the sequencer is the only one that sets the sequence number the number is preserved within the message.
\subsection{Integrity}
\subsection{Validity}
\subsection{Agreement}
\subsection{Integrity}
\section{Ordering}
\subsection{Total order}
  The sequencer is the one that takes care of the ordering. No one else is allowed to set a sequence number.
  If a message is sent, it has to be passed to the sequencer, get a sequence number, and then it is delivered back
  by a broadcast. When it is recieved it is not delivered, it has to be put to a queue untill the message with the
  next sequence number if found. Hence there is only one node that is allowed to set sequence numbers, and the 
  sequence numbers is unique, all the nodes will get a message with a sequence number.
\subsection{Causal order}
\section{Fault tolerance}
\section{Conclusion}
\end{document}


